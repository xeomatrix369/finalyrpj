\chapter{Discussion and Conclusion}

We propose two components essential for autonomous systems that interact with their surrounding environments. These are in fact two of the key computer vision problems that have been attempted for a long time. 

Firstly, we present an end-to-end system capable of performing multi-object tracking by combining a range of advances in object detection and reidentification along with our novel architectures and loss functions. Further, we work on a novel step by building a separate LSTM branch to estimate the similarity feature map for the next time step of a given track. The Siamese Networks may be viewed as a two-step version of our extension, whereas this replacement with an LSTM is more of a generalized version capable of generating a better feature set. The key expectation with this addition is the overcoming of identity switches and lost tracks in the case of occlusions. Appearance features tend to change significantly during an occlusion, especially when an object undergoes rotations, and our extension overcomes this by modeling the appearance changing pattern over time. 

Thereafter, we proposed a probabilistic graphical model based framework for panoptic segmentation. Our CRF model with two different kinds of random variable, named Bipartite CRF or BCRF, is capable of optimally combining the predictions from a semantic segmentation model and an instance segmentation model to obtain a good panoptic segmentation. We use different energy functions in our BCRF to encourage the spatial, appearance, and instance-to-semantic consistency of the panoptic segmentation. An iterative mean field algorithm was then used to find the panoptic labeling that approximately maximizes the conditional probability of the labeling given the image. We further showed that the proposed BCRF framework can be used as an embedded module within a deep neural network to obtain superior results in panoptic segmentation.

\section{Principles, Relationships and Generalizations inferred from results}
As depicted in the results section, our tracker has shown improvements basically in relation to MOT evaluation metrics. The improvements presented based on the KITTI dataset (which has 9 separate classes) shows how our system has generalized multi class tracking without the need for training separate computationally expensive re-identification networks. MOT16 contains data belonging to the pedestrian class only but the movement of objects in this class is subjugated to more occlusions and random movements compared to the KITTI dataset. The improvement of MOTA over MOT16 dataset indicates signs that our system handles occlusions better. It is also evident not only through the dataset statistics but also through the visual online videos that our system has less number of lost tracks in the middle of a certain scenario.

In panoptic segmentation, the results depict the principle analysis that bipartite conditional random fields propose an improved labeling in both semantic as well as instance domains where initial unary potentials for semantic and instance identities are taken from unary classifiers that are state of the art systems at present. The results also show that the cross potential component of the aggregated energy function that is being minimized during an inference has effects beyond rest of the energy function with both semantic component and instance component separately. The improvements observed in the Panoptic Quality are also visually consistent with intuition that stray patches of the final output have mostly been removed and the edges of objects have been smoothened. The final output when split and analyzed semantically and instance wise; the qualitative results present the consistency and clarity in comparison to the unary classifier outputs.


\section{Problems and Exceptions to the Generalizations}
The results show that MOTP of our tracker is considerably low in MOT16 dataset in comparison to other systems. This indicates that the LSTM network is unable to handle rapid variations of the bounding box parameters. This is to be expected as the bounding box variations in datasets such as MOT16 is extremely chaotic in cases where the pedestrian is rotating while walking and moving in general. This is also due to the morphological changes of the moving body specifically a bounding box is not an ideal interpretation of the object. The hand gesture changes are also changing the bounding box co-ordinates of the object considered. However this complication does not arise for the cases where automobiles are considered. It was also observed that system has higher performance in time domain when automobile motion is considered.

The system implemented for panoptic segmentation through the aggregation of two separate heads built for semantic and instance segmentation having state of the art accuracy builds up a compatibility matrix that compares the class wise cross compatibility of the instance and semantic classes. This learns the entry matrix elements from the dataset. However if the dataset is biased for say person class (As in the case of Pascal VOC); there is a tendency of having arbitrarily high compatibility which is dataset dependent. This can be avoided by using large datasets which are robust however that training task requires considerable computational resources. 


\section{Agreements/Disagreements with previously published work}
The results agree with recently published systems such as Deep SORT \cite{DeepSiam:deepSort}. It is expected that as ML decreases when MOTA increases as it reduces the number of false negatives considerably. This correlation is depicted in our results.
However the experiments that had been run basing the data association LSTM network did not turn successful as presented in \cite{DeepSiam:MilanL0RS16}. However in \cite{DeepSiam:MilanL0RS16} it describes as a network not promised to have high accuracy but possesses higher frame rate in comparison to the accuracy. As a result, the lack of data association capability and the retarded smoothness in convergence could be expected when single module is isolated from the aforementioned network and tried to train starting from Xavier initialization.

We were able to replicate the recent most state of the art systems to obtain the unary classifications on image segmentation. The approach followed by our system agrees with the work published by authors in \cite{Zhen_ICCV15_CRFRNN} for refining the output of a single head semantic segmentation network using conditional random fields. Our system was integrated on top of a state of the art system presented in [76]. We used the loss function presented in \cite{Anurag17} for training.  
