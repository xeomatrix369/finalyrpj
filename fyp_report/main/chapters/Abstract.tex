\chapter*{Abstract}
\addcontentsline{toc}{chapter}{Abstract}

\begin{center}
	\vspace{5mm}
	\MakeUppercase{\textbf{Realtime Multi-Object Tracking and Pixelwise Segmentation}}\\
	\vspace{5mm}
	Group Members: \memberA, \memberB, \memberC, \\ \memberD \\
	\vspace{5mm}
	Supervisors: \supervisorA, \supervisorB \\
	\vspace{5mm}
\end{center}

\noindent Keywords: Vision, Perception, Detection, Tracking, Panoptic Segmentation, Siamese Network, Conditional Random Field, Recurrent Neural Network, Autonomous Systems. \\

Bleeding-edge technological pursuits ranging from self-guided robots at the research stage to mass scale industrial applications such as augmented reality, intelligent security systems and self-driving vehicles heavily rely on perception through vision. Vision based perception of the environment in autonomous systems extensively use object detection, segmentation and tracking as fundamental components. Despite the recent advancements in deep learning-based object detection on monocular images, several highly publicized accidents involving self-driving vehicles and critical failures in monitoring systems highlight the need for significant further improvement on real-time tracking systems in practice. We identify two such key areas with room for improvement and introduce two separate novel frameworks to tackle each problem. 

We observe that trackers often perform poorly in object dense situations where occlusions and crossovers are prevalent. We identify that in order to perform better in these scenarios both appearance and motion information should be incorporated. Siamese networks have recently become highly successful at appearance based single object tracking while Recurrent Neural Networks (RNNs) have started dominating motion-based tracking. Our work focuses on combining Siamese networks and RNNs to exploit both (temporally varying) appearance and motion information to build a robust framework that can also operate in real-time. We further explore heuristics-based constraints for tracking in the Bird’s Eye View Space for efficiently exploiting 3D information.

Our segmentation approach is based on one of the most overwhelming problems in current vision community that has full scale perception on the image, known as panoptic segmentation where pixel level identification of the entire image is done with both semantic and instance information thus integrating object classes (thing classes having countable instance segmentation) and back-ground classes (stuff, amorphous) in a single frame. We tackle the panoptic segmentation problem with a conditional random field (CRF) model. At each pixel, the semantic label and the instance label should be compatible and spatial and color consistency of the labeling has to be preserved (similar looking neighboring pixels should have the same semantic label and the instance label). To tackle this problem, we propose a fully differentiable model named Bipartite CRF (BCRF) which can be included as a trainable first class citizen in a deep network.
