\chapter*{Abstract}
\addcontentsline{toc}{chapter}{Abstract}

\begin{center}
	\vspace{5mm}
	\MakeUppercase{\textbf{PLL Transistor Level Implementation}}\\
	\vspace{5mm}
	Group Members: \memberA, \memberB, \memberC, \\ \memberD \\
	\vspace{5mm}
	Guide: \guideA, \guideB \\
	\vspace{5mm}
\end{center}

\noindent \textbf{Keywords:} \\

In this project we have endevoured to Design transistor level circuit to design a pll at technology of 180nm 

A PLL is closed loop frequency system that locks phase of an output signal to an input reference signal ie. the ouput signal is an function of the input signal and the term lock hear means. 
PLL is used in many applications such as clock generation, clock recovery, frequency synthesis, demodulation and so on. The PLL is a feedback system that compares the phase of the output signal with the phase of the input reference signal and adjusts the output signal to minimize the phase difference. The PLL consists of three main components: a phase detector, a loop filter, and a voltage-controlled oscillator (VCO). The phase detector compares the phase of the input reference signal with the phase of the output signal and generates an error signal that represents the phase difference. The loop filter processes the error signal to remove high-frequency noise and generates a control voltage that is used to adjust the frequency of the VCO. The VCO generates an output signal whose frequency is proportional to the control voltage.

The design is carried out using LTspice with focus to demonstrate proof of concept 