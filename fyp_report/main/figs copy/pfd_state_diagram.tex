\begin{figure} [!h]
    \centering
    \resizebox{0.8\textwidth}{!}{%
    

% \begin{tikzpicture}[->, >=stealth', auto, node distance=3.5cm, semithick]
% \tikzstyle{every state}=[draw=black, thick, fill=gray!10, minimum size=1.2cm]

% % Nodes
% \node[state] (S1) {UP=0\\DN=1};
% \node[state, right of=S1] (S2) {UP=0\\DN=0};
% \node[state, right of=S2] (S3) {UP=1\\DN=0};

% % Transitions
% \path 
% (S1) edge[loop left] node[align=center]{FB ↑} (S1)
% (S1) edge[bend left] node{REF ↑} (S2)
% (S2) edge[bend left] node{FB ↑} (S1)
% (S2) edge[bend left] node{REF ↑} (S3)
% (S3) edge[bend left] node{FB ↑} (S2)
% (S3) edge[loop right] node[align=center]{REF ↑} (S3);

% \end{tikzpicture}
\begin{tikzpicture}[node distance=3cm]
    \node[state] (q1) {a};
    \node[state, right of =q1] (q2) {b};
    \node[state, right of = q2] (q3) {c};
    % \node[state] (q2) [side of= q1] {locked};
\end{tikzpicture}
    
    }%
    \caption{PFD state diagram}
    \label{fig:pfd_state_diagram}
    % \vspace{-0.5cm}
\end{figure}