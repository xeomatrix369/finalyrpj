%% Author: Mohammed Hamza
%% Date:
%% Description: Generic and customizable thesis/dissertation template.
%% URL: https://github.com/Bekt/thesis-template

\documentclass[12pt,letterpaper]{report}
\usepackage[left=1.5in, right=1.0in, top=1.0in, bottom=1.0in]{geometry}

% Check if with latex or pdflatex.
\ifx\pdftexversion\undefined
  \usepackage[dvips]{graphicx}
\else
  \usepackage[pdftex]{graphicx}
\fi

\usepackage{setspace}
\usepackage{titlesec}
\usepackage{enumerate}
\usepackage{enumitem}
\usepackage{amsmath}
\usepackage{amssymb}
\usepackage{fancyvrb}
\usepackage{float}
\usepackage{hyperref}
\usepackage{svg}
\usepackage{fancyhdr} % Required for fancy page styles
\usepackage{xcolor} % Required for defining custom colors
\usepackage{listings} % Required for code listings
\usepackage{tabulary}
\usepackage{multicol}
\usepackage{times}
\usepackage{epsfig}
\usepackage{tikz}
\usepackage{amsmath}
\usepackage{amsfonts}
\usepackage{amssymb}
\usepackage{circuitikz}
% \usepackage{sagetex}
% \usepackage{graphicx}
\usetikzlibrary{shapes, arrows.meta,arrows, positioning,chains, quotes}

% variables definition
\newcommand{\memberA}{Mohammed Hamza}
\newcommand{\memberB}{Nanda kishore}
\newcommand{\memberC}{def}
\newcommand{\memberD}{xyz}
\newcommand{\memberE}{xyz}
\newcommand{\indexA}{U03NM21T043044}
\newcommand{\indexB}{150360A}
\newcommand{\indexC}{150273J}
\newcommand{\indexD}{150504V}
\newcommand{\indexE}{150504V}
\newcommand{\guideA}{Dr BPH}
\newcommand{\guideB}{Dr HMV}
\input{defs}

% Table of contents.
% \setcounter{secnumdepth}{3}
% \setcounter{tocdepth}{3}

% % Chapter title format, section format, subsection format.
% \titleformat{\chapter}
%   {\normalfont\Large\bfseries}{\thechapter}{1em}{}
% \titleformat{\section}
%   {\normalfont\Large\bfseries}{\thesection}{1em}{}
% \titleformat{\subsection}
%   {\normalfont\bfseries}{\thesubsection}{1em}{}
% 
% \titleformat{\section}[block]{\centering\Large\bfseries}{}{0pt}{} 
% \titlespacing*{\section}{0pt}{\baselineskip}{\baselineskip}
% % Chapter, section, subsection title spacings.
% \titlespacing*{\chapter}{0pt}{0ex plus 1ex minus .2ex}{2.3ex plus .2ex}
% \titlespacing*{\section} {0pt}{1ex plus 1ex minus .2ex}{1.5ex plus .2ex}
% \titlespacing*{\subsection} {0pt}{2ex plus 1ex minus .2ex}{1ex plus .2ex}

% renew counter for section numbering


% Spacing after a caption.
\setlength{\belowcaptionskip}{-15pt}

% % Spacing before a footer.
% \setlength{\skip\footins}{0.5cm}
\geometry{a4paper, margin=1in} 

\pagestyle{fancy}
\fancyhf{}
\fancyhead[L]{PLL implementation}


\pagestyle{fancy}
\fancyhf{} % Clear all header and footer fields
\fancyhead[L]{PLL}
\fancyhead[R]{\makebox[\textwidth][r]{\textbf{\nouppercase{\leftmark}}}} % Right-alig

\fancyfoot[L]{\textbf{Department of ECE}} % Left side: Department name
\fancyfoot[C]{\textbf{Jan 2025 – Apr 2025}} % Center: Semester dates
\fancyfoot[R]{\textbf{Page \thepage}} % Right side: Page number
\renewcommand{\footrulewidth}{0.4pt}
% Define a custom page style for section pages
\fancypagestyle{chapterstyle}{
  \fancyhf{} % Clear all header and footer fields
  \renewcommand{\headrulewidth}{0pt} 
  \fancyfoot[L]{\textbf{Department of ECE}} % Left side: Department name
  \fancyfoot[C]{\textbf{Jan 2025 – Apr 2025}} % Center: Semester dates
  \fancyfoot[R]{\textbf{Page \thepage}} % Right side: Page number
}

% Apply custom style to section pages
% \makeatletter
% \preto\section{\thispagestyle{sectionstyle}}
% \makeatother

% \makeatletter
% \preto\chapter{\thispagestyle{chapterstyle}}
% \makeatother

% No header on the first page of a section




% \renewcommand{\section}{
%     \clearpage % Start each section on a new page
%     \pagestyle{plain} % Remove header
%     \oldsection
%     \thispagestyle{plain} % Ensure no header on this first page
%     \pagestyle{fancy} % Resume normal header style
% } have to check thisfkdljal;dskfa;sdkl;falsd;f

\definecolor{light-gray}{gray}{0.95} % Define the light-gray color
\lstset{
  basicstyle=\ttfamily\small,
  basicstyle=\ttfamily\small,
  keywordstyle=\color{blue}\bfseries,
  stringstyle=\color{red},
  commentstyle=\color{gray},
  morekeywords={int, return, Serial, if, else}, % Add any other keywords specific to your code
  numbers=left,
  numberstyle=\tiny\color{gray},
  stepnumber=1,
  breaklines=true,
  captionpos=b,
  tabsize=2,
  showspaces=false,
  showstringspaces=false,
  showtabs=false,
  breakatwhitespace=true,
  backgroundcolor=\color{light-gray}, % Set background color
}

\begin{document}

  \onehalfspacing
  \begin{titlepage}
  \centering % Center all text
	\vspace*{\baselineskip} % White space at the top of the page
	
	\rule{\textwidth}{1.6pt}\vspace*{-\baselineskip}\vspace*{2pt} % Thick horizontal line
	\rule{\textwidth}{0.4pt}\\[\baselineskip] % Thin horizontal line
	{\LARGE PLL Transistor Level Implementation} \\
	
	\rule{\textwidth}{0.4pt}\vspace*{-\baselineskip}\vspace{3.2pt} % Thin horizontal line
	\rule{\textwidth}{1.6pt}\\[\baselineskip] % Thick horizontal line
	
	\scshape % Small caps
	
	\vspace*{1\baselineskip} % Whitespace between location/year and editors
  \vspace{1 in}
  \begin{center}
    {\large{\textsl{
				{A \\ 
					  Project Report \\ %\underline{•}
					Submitted in Partial Fulfillment of the Bangalore University\\
					for the Degree \\ 
					of\\
					\large \bf Bachelor of Technology \\
                    in\\
                    \large \bf Electronics and \\
                    \large \bf Communication Engineering 
	  }}}}\\
    \vspace{50mm}

    
    
%	Supervisor:	 	\hfill  	Group Members: \\
%    \supervisorA \hfill 	\memberA  - \indexA \\
%    \supervisorB \hfill 	\memberB  - \indexB \\
%    					  \hfill 	 \memberC  - \indexC \\
%   						  \hfill 	 \memberD  - \indexD \\
    
    \vspace{40mm}
    February, 2020\\

  \end{center}
\end{titlepage}



  \singlespacing
  
    % Roman numeral numbering until introduction.
  \pagestyle{plain}
  \pagenumbering{arabic}
  
  % \include{chapters/Signatures}
  % \include{chapters/Declaration}
  % \chapter*{Declaration by Supervisor}
 \addcontentsline{toc}{chapter}{Declaration by Supervisor}

\begin{flushleft}
	I/We have supervised and accepted this dissertation for the submission of the degree. \\

	\vspace{15mm}
	
	{\makebox[6.5cm]{\dotfill}} \hfill {\makebox[5cm]{\dotfill}}  \\ 
	\supervisorA \hfill Date \\
	
	
	\vspace{15mm}
	
	{\makebox[6.5cm]{\dotfill}} \hfill {\makebox[5cm]{\dotfill}}  \\ 
	\supervisorB \hfill Date \\
	
\end{flushleft}
 
  \chapter*{Abstract}
\addcontentsline{toc}{chapter}{Abstract}

\begin{center}
	\vspace{5mm}
	\MakeUppercase{\textbf{Realtime Multi-Object Tracking and Pixelwise Segmentation}}\\
	\vspace{5mm}
	Group Members: \memberA, \memberB, \memberC, \\ \memberD \\
	\vspace{5mm}
	Supervisors: \supervisorA, \supervisorB \\
	\vspace{5mm}
\end{center}

\noindent Keywords: Vision, Perception, Detection, Tracking, Panoptic Segmentation, Siamese Network, Conditional Random Field, Recurrent Neural Network, Autonomous Systems. \\

Bleeding-edge technological pursuits ranging from self-guided robots at the research stage to mass scale industrial applications such as augmented reality, intelligent security systems and self-driving vehicles heavily rely on perception through vision. Vision based perception of the environment in autonomous systems extensively use object detection, segmentation and tracking as fundamental components. Despite the recent advancements in deep learning-based object detection on monocular images, several highly publicized accidents involving self-driving vehicles and critical failures in monitoring systems highlight the need for significant further improvement on real-time tracking systems in practice. We identify two such key areas with room for improvement and introduce two separate novel frameworks to tackle each problem. 

We observe that trackers often perform poorly in object dense situations where occlusions and crossovers are prevalent. We identify that in order to perform better in these scenarios both appearance and motion information should be incorporated. Siamese networks have recently become highly successful at appearance based single object tracking while Recurrent Neural Networks (RNNs) have started dominating motion-based tracking. Our work focuses on combining Siamese networks and RNNs to exploit both (temporally varying) appearance and motion information to build a robust framework that can also operate in real-time. We further explore heuristics-based constraints for tracking in the Bird’s Eye View Space for efficiently exploiting 3D information.

Our segmentation approach is based on one of the most overwhelming problems in current vision community that has full scale perception on the image, known as panoptic segmentation where pixel level identification of the entire image is done with both semantic and instance information thus integrating object classes (thing classes having countable instance segmentation) and back-ground classes (stuff, amorphous) in a single frame. We tackle the panoptic segmentation problem with a conditional random field (CRF) model. At each pixel, the semantic label and the instance label should be compatible and spatial and color consistency of the labeling has to be preserved (similar looking neighboring pixels should have the same semantic label and the instance label). To tackle this problem, we propose a fully differentiable model named Bipartite CRF (BCRF) which can be included as a trainable first class citizen in a deep network.

  \chapter*{Dedication}
\addcontentsline{toc}{chapter}{Dedication}

\begin{center}
	\vspace{100mm}
	To our families, friends, Guide, and all others that supported us in this work. \\
\end{center}


  \chapter*{Acknowledgements}
\addcontentsline{toc}{chapter}{Acknowledgements}

\vspace{10mm}
We express our sincere gratitude to our supervisors, \guide  and \hod, for their endless guidance, support, and commitment towards the success of this project. We would also like to thank FiveAI, UK for providing resources and support in conducting certain experiments for this project. 



  
  \onehalfspacing
  \tableofcontents
  \addcontentsline{toc}{chapter}{Contents}
  \pagebreak
  
  \listoffigures
  \pagebreak
  
  \listoftables
  \pagebreak
  
    \chapter*{Acronyms and Abbreviations}
\addcontentsline{toc}{chapter}{Acronyms and Abbreviations}
\vspace{5mm}
\begin{tabular}{ll}
\textbf{PLL}  & Phase Locked Loop \\
\textbf{VCO}  & Voltage Controlled Oscillator \\
\textbf{PID}  & Proportional-Integral Derivative \\
\textbf{PD}   & Proportional Derivative \\
\textbf{PFD}  & Phase Frequency Detector \\
\textbf{TSMC} & Taiwan Semiconductor Manufacturing Company \\
\textbf{OTA}  & Operational Transconductance Amplifier \\
\textbf{FD}   & Frequency Divider \\
\textbf{LF}   & Loop Filter \\
\textbf{CP}   & Charge Pump \\
\end{tabular}

    \include{chapters/Symbols}
  

  % Change the numbering back to normal.
  \pagestyle{plain}
  \pagenumbering{arabic}
  \setcounter{page}{1}

  \onehalfspacing
  \chapter{Introduction}
 A PLL is a feedback system that includes a VCO, phase detector, and low pass filter within its loop.Its purpose is to force the VCO to replicate and track the frequency and phase at the input when in lock. The PLL is a control system allowing one oscillator to track with another.
\begin{equation}
    \label{eq:pll_1}
    \phi_{\text{out}}(t) = \phi_{\text{in}}(t) + \text{const.}
\end{equation}
\begin{figure}[h]
    % \vspace{-0.7cm}
      \centering
      \includegraphics[width=0.5\textwidth]{figs/Phase_locked_loop.png}      \vspace{-0.3cm}
      \caption[]{ Simple analog phase locked loop}
      \label{fig:pll_1}
    % \vspace{0.5cm}
\end{figure}    
\section{Motivation}
Our team chose to work on the PLL project to gain practical knowledge about the fundamentals of phase-locked loops and their applications in modern electronics. We were particularly interested in the design and implementation of a PLL circuit at the transistor level to understand the underlying principles of PLL operation and the challenges involved in designing such circuits. We were motivated by the opportunity to work with simulation tools and techniques, such as LTspice, to analyze and optimize.

\section{Brief History and Applications of PLL}
\begin{itemize}
    \item The PLL was invented in 1932 by Harold Stephen Black, an engineer at Bell Labs.
    \item The original PLL was used in telephone systems to eliminate noise and improve the quality of voice signals.
    \item Since then, the PLL has been widely used in various applications, including:
    \begin{itemize}
        \item Clock generation: Ensuring that all components operate synchronously.
        \item Frequency synthesis: Generating a range of frequencies from a single reference frequency.
        \item Demodulation: Used in communication systems like FM radio and digital communication systems.
        \item Data recovery: Recovering clock signals from data streams for proper synchronization.
        \item Jitter reduction: Improving the performance of high-speed digital circuits.
        \item Phase alignment: Aligning the phase of multiple signals for proper timing and synchronization.
        \item Frequency modulation: Modulating signals in communication systems like FSK and PSK.
        \item Signal conditioning: Filtering and conditioning signals to improve quality and reduce noise.
        \item Clock recovery: Ensuring proper synchronization between transmitting and receiving devices.
    \end{itemize}
\end{itemize}

\section{Objective of the Project}
\begin{itemize}
    \item To design a PLL circuit at the transistor level using LTspice, with specific characteristics and specifications.
    \item Design targets:
    \begin{itemize}
        \item Reference Frequency = 20 MHz
        \item Output Frequency = 2.4 GHz
        \item Power Consumption = 2 mW
        \item Deterministic Jitter = 10 ps, pp
        \item Random Jitter = 2 ps, rms
        \item Supply Voltage = 1 V
    \end{itemize}
    \item Carry out simulations to verify the performance of the designed PLL circuit.
    \item Analyze the results and compare them with the design targets.
    \item Identify the charachteristics of the PLL circuit for example:
    \begin{itemize}
        \item Phase noise
        \item Jitter
        \item Lock time
        \item Frequency stability
        \item Power consumption
        \item Output waveform
        \item Phase margin
        \item Loop bandwidth
        \item frequecy capture range
    \end{itemize}       
    \item Document the design process, challenges faced, and lessons learned during the project.
\end{itemize}
  \chapter{Literature Review}

Phase-Locked Loops (PLLs) are critical components in modern electronic systems, widely used for frequency synthesis, clock generation, and signal synchronization. This section reviews the foundational concepts, advancements, and applications of PLLs as presented in the literature.

\section{Fundamentals of PLLs}
% The PLL is a feedback control system that synchronizes the phase of an output signal with a reference signal. Early works, such as those by Gardner \cite{gardner2005phaselock}, laid the theoretical foundation for PLL design, focusing on the mathematical modeling and stability analysis of the loop.

\section{Types of PLLs}
Several types of PLLs have been developed to cater to different applications:
\begin{itemize}
    \item \textbf{Analog PLLs (APLLs):} These are the traditional PLLs that use analog components such as voltage-controlled oscillators (VCOs) and phase detectors.
    \item \textbf{Digital PLLs (DPLLs):} With advancements in digital technology, DPLLs have gained popularity due to their robustness and programmability.
    \item \textbf{All-Digital PLLs (ADPLLs):} These PLLs eliminate analog components entirely, offering better integration in digital systems.
\end{itemize}

\section{Applications of PLLs}
PLLs are employed in a wide range of applications:
\begin{itemize}
    \item \textbf{Communication Systems:} PLLs are used for carrier recovery, clock recovery, and frequency synthesis in wireless and wired communication systems.
    \item \textbf{Microprocessors:} Modern processors use PLLs for clock generation and synchronization to achieve high-speed operation.
    \item \textbf{Power Electronics:} PLLs are utilized in grid synchronization for renewable energy systems and motor control applications.
\end{itemize}

\subsection{Recent Advancements}
% Recent research has focused on improving the performance of PLLs in terms of phase noise, power consumption, and lock time. Techniques such as adaptive loop filters, machine learning-based optimization, and low-power design methodologies have been explored to enhance PLL performance \cite{recent_pll_advancements}.

\subsection{Challenges and Future Directions}
Despite significant progress, challenges remain in designing PLLs for ultra-low power applications, high-frequency operation, and integration in advanced semiconductor technologies. Future research is expected to address these challenges by leveraging emerging technologies such as quantum computing and advanced materials.

% \bibliographystyle{IEEEtran}
% \bibliography{references} 
  \chapter{Methodology/Design/Implementation}
\label{chapter:method}
\section{Phase Locked Loop System Overview}
\label{sec:pll_overview}
A Phase-Locked Loop (PLL) is a negative feedback control system circuit. As the name implies, the purpose of a PLL is to generate a signal whose phase matches that of a reference signal. This is achieved through multiple iterations of comparing the reference and feedback signals (ref fig:\ref{fig:pll_1}). The overall goal of the PLL is to align the phases of the reference and feedback signals—this is referred to as the lock mode. Once locked, the PLL continues to compare the two signals, but since they are in lock mode, the PLL output remains constant. \\
\textbf{A basic PLL consists of four main components:}
\begin{enumerate}
	\item Phase Detector or Phase Frequency Detector (PD or PFD)
	\item Charge Pump (CP)
	\item Low Pass Filter (LPF)
	\item Voltage-Controlled Oscillator (VCO)
\end{enumerate}
The Phase Frequency Detector (PFD) measures the phase difference between the reference and feedback signals. If a phase difference exists, it generates synchronized “up” or “down” signals to the charge pump and low pass filter. If the error signal from the PFD is an “up” signal, the charge pump adds charge to the LPF capacitor, increasing the control voltage, \( V_{\text{cntrl}} \). Conversely, if the error signal is a “down” signal, the charge pump removes charge from the LPF capacitor, decreasing \( V_{\text{cntrl}} \).\tikzstyle{block} = [draw, rectangle]
\tikzstyle{input} = [coordinate]
\tikzstyle{output} = [coordinate]
\tikzstyle{pinstyle} = [pin edge={to-,thin,black}]
\begin{figure}[h]
    \centering
    \begin{tikzpicture}[
      block/.style={draw, thick, minimum width=3cm, minimum height=2.5cm, align=center},
      line/.style={-Stealth, thick},
      node distance=1.5cm and 1.5cm
    ]
    
    % Blocks
    \node[input,name=inputref] (input) {fref};
    \node[block, right=2cm of input] (pd) {PD};
    \node[block, right of=pd] (cp) {CP};
    \node[block, right of=cp] (vco) {VCO};
    \node[block, below of=cp , node distance=1cm] (lpf) {LPF};
    
% Signals
\coordinate[left=1.5cm of pd] (inputref);
\coordinate[below=0cm of inputref] (inputfb);
\coordinate[right=1.5cm of vco] (output);
    
    % Connections
    % \draw [->] (inputref) -- node {$fref$} (pd);
    % \draw[--] (inputref) node[left] {$f_{ref}$} -- (pd);
    % \draw[--] (inputfb) node[left] {$f_{fb}$} -- (pd);
    \draw[line] ( inputref) -- node[above] {$f\_ref$} (pd);
    \draw[line] (inputfb) -- node[above] {$f\_fb$} ([yshift=2cm]pd);
    % \draw[--] (pd) -- (cp);
    % \draw[line] (cp) -- node[above] {$V_{cntrl}$} (vco);
    % \draw[line] (cp) |- (lpf);
    % \draw[line] (lpf) -| (vco);
    
    % \draw[line] (vco) -- node[above] {$f_{out}$} (output);
    % \draw[line] (output) |- ++(0,-2) -| (inputfb);
    
    \end{tikzpicture}
    \caption{PLL Block Diagram}
\end{figure}\\
The control voltage \( V_{\text{cntrl}} \) serves as the input to the VCO. The LPF is essential for allowing only DC signals into the VCO and for storing the charge from the CP. The VCO adjusts the feedback signal's frequency based on the error generated by the PFD. If the PFD generates an “up” signal, the VCO speeds up the feedback signal. Conversely, if a “down” signal is generated, the VCO slows it down. The output of the VCO is then fed back to the PFD to recalculate the phase difference, thereby creating a closed-loop frequency control system.

\subsection{Phase Detector}
A phase detector is a circuit that detects the difference in phase between its two input
signals. An example of a basic phase detector is the XOR gate. It produces error pulses on both
falling and rising edges.

\begin{figure}[!h]
\centering
\resizebox{0.3\textwidth}{!}{%
\begin{circuitikz}
\tikzstyle{every node}=[font=\huge]
% \draw (11.75,11.25) to[short] (12.25,11.25);
% \draw (11.75,10.75) to[short] (12.25,10.75);
\draw (12.25,11.25) node[ieeestd xor port, anchor=in 1, scale=1](port){} (port.out) to[short] (14.25,11);
\node [font=\LARGE] at (11.25,11.5) {$\phi_{\text{ref}}$};
\node [font=\LARGE] at (11.25,10.75) {$\phi_{\text{vco}}$};
\end{circuitikz}
}%
\label{fig:xor_gate}
\caption{}
\end{figure}
\subsubsection{Phase Frequency Detector}
A phase frequency detector (PFD) is a circuit that detects the difference in phase and frequency between its two input signals. PFD is a more advanced version of the basic phase detector. It can detect both phase and frequency differences, making it more suitable for applications where the input signals may have different frequencies. The PFD generates an error signal that is proportional to the phase and frequency difference between the two input signals. This error signal is then used to control the charge pump and low pass filter in the PLL.
\begin{figure}[!h]
    \centering
    \resizebox{0.8\textwidth}{!}{%
    \includegraphics{figs/pfd_block.png}
    }%
    \caption{(a) Conceptual PFD operation, (b) case of input phase difference, and (c) case of input frequency difference}
    \label{fig:pfd_block}
\end{figure}

PFD state diagram is shown in Figure \ref{fig:pfd_state_diagram}. 
The PFD circuit is a sequential circuit which produces UP and DOWN outputs which inform the CP to charge or discharge the loop
filter depending on the phase difference between the two input signals of PFD. Depending on which signal becomes high first, the REF signal or the FB signal in Figure \ref{fig:pfd_state_diagram}, the circuit goes either to state 1 or to state 2. Supposing the circuit goes to state 1, if the FB signal becomes high, the circuit goes back to state 0. The longer it takes the FB to transit to 1, the longer will UP high signal remain high. This wider pulse on the other hand serves as a control voltage for the CP. The bigger the width of the UP signal, the greater the input voltage to the VCO will be, increasing the frequency of the output signal. Basically, the circuits let the CP know if the phase of the Frequency Divider (FD) circuit is higher or lower than some reference signal and helps to discharge or charge the node right after the CP circuit. If the frequency of the input signal is higher than the frequency of the output signal, the circuit will only go from state 0 to state 1 and will keep charging up the LF. On the other scenario, if the input frequency is lower than the output frequency, the output of the CP will be discharged leading to reduced frequencies on the VCO circuit.

\begin{figure} [!h]
    \centering
    \resizebox{0.8\textwidth}{!}{%
    

% \begin{tikzpicture}[->, >=stealth', auto, node distance=3.5cm, semithick]
% \tikzstyle{every state}=[draw=black, thick, fill=gray!10, minimum size=1.2cm]

% % Nodes
% \node[state] (S1) {UP=0\\DN=1};
% \node[state, right of=S1] (S2) {UP=0\\DN=0};
% \node[state, right of=S2] (S3) {UP=1\\DN=0};

% % Transitions
% \path 
% (S1) edge[loop left] node[align=center]{FB ↑} (S1)
% (S1) edge[bend left] node{REF ↑} (S2)
% (S2) edge[bend left] node{FB ↑} (S1)
% (S2) edge[bend left] node{REF ↑} (S3)
% (S3) edge[bend left] node{FB ↑} (S2)
% (S3) edge[loop right] node[align=center]{REF ↑} (S3);

% \end{tikzpicture}
\begin{tikzpicture}[node distance=3cm]
    \node[state] (q1) {a};
    \node[state, right of =q1] (q2) {b};
    \node[state, right of = q2] (q3) {c};
    % \node[state] (q2) [side of= q1] {locked};
\end{tikzpicture}
    
    }%
    \caption{PFD state diagram}
    \label{fig:pfd_state_diagram}
    % \vspace{-0.5cm}
\end{figure}

This PFD circuit senses the 0 to 1 transitions of VREF and VFB and produces UP and DOWN signals depending on which of the input signal rises quicker. Theoretically, this is a very easy circuit to be build and it should work very close to the ideal circuit.
One of the problems that is faced in real PFD blocks is the dead-zone of the PFD which happens for very small phase differences. The block diagram of PFD is shown in \ref{fig:pfd_dff_blk}
\begin{figure}[h]
	\centering
	\includegraphics[width=0.5\textwidth]{figs/pfd_dff_blk.png}
	% \vspace{-0.3cm}
	\caption{PFD circuit using D flip-flops}
	\label{fig:pfd_dff_blk}
	\vspace{0.5cm}
\end{figure}
\subsection{Nanda biceps Pump}
The charge pump is a circuit that converts the error signal from the phase detector into a control voltage for the VCO. The charge pump consists of two switches and a capacitor. The switches are controlled by the error signal from the phase detector. When the error signal is high, the switch connects the capacitor to the power supply, charging it. When the error signal is low, the switch connects the capacitor to ground, discharging it. The control voltage for the VCO is taken from the capacitor.
\begin{figure}[!ht]
    \centering
    \resizebox{0.2\textwidth}{!}{%
    \begin{circuitikz}
    \tikzstyle{every node}=[font=\large]
    
    
    \draw [ line width=0.6pt](10.75,16) to[american current source,l={ \large Icp}] (10.75,14);
    \draw [ line width=0.6pt](10.75,11.75) to[american current source,l={ \large Icp}] (10.75,9.5);
    \draw [line width=0.6pt](10.75,11.75) to[normal open switch] (10.75,12.5);
    \draw [line width=0.6pt](10.75,12.5) to[normal open switch] (10.75,14);
    \draw [line width=0.6pt](10.75,9.75) to (10.75,9.5) node[ground]{};
    \draw [ line width=0.6pt](10.25,16) to[short] (11.25,16);
    \draw [ line width=0.6pt](10.75,12.75) to[short, -o] (12,12.75) ;
    \draw [ line width=0.6pt](10.75,12) to[short, -o] (9.25,12) ;
    \draw [ line width=0.6pt](10.75,13.25) to[short, -o] (9.25,13.25) ;
    \node [font=\large] at (12.5,13) {Iout};
    \node [font=\large] at (8.25,13.25) {Up};
    \node [font=\large] at (8.5,12) {Down};
    \end{circuitikz}
    }%
    
    \label{fig:my_label}
\end{figure}

The CP circuit should ideally behave as in Table~
\ref{tab:cp_function}.

\begin{table}[h!]
\centering
\begin{tabular}{|c|c|c|}
\hline
\textbf{UP} & \textbf{DOWN} & \textbf{IOUT} \\
\hline
0 & 0 & 0 \\
0 & 1 & $-\text{I}_{\text{CP}}$ \\
1 & 0 & $+\text{I}_{\text{CP}}$ \\
1 & 1 & 0 \\
\hline
\end{tabular}
\caption{Charge Pump Functionality.}
\label{tab:cp_function}
\end{table}

\subsection{Low Pass Filter}
The low pass filter is a circuit that removes high frequency noise from the control voltage generated by the charge pump. The low pass filter consists of a resistor and a capacitor. The resistor limits the current flowing into the capacitor, while the capacitor stores the charge. The output of the low pass filter is a smooth control voltage that is fed to the VCO.
% \input{figs/lpf.tex}


\subsection{Voltage Controlled Oscillator}
The voltage controlled oscillator (VCO) is a circuit that generates an output signal whose frequency is proportional to the control voltage. The VCO consists of a transistor and a capacitor. The transistor is biased by the control voltage, which determines its operating frequency. The output of the VCO is fed back to the phase detector to complete the PLL loop.
\begin{equation}
	\label{eq:vco_char}
	f_{out} = K_{vco} * V_{in} + f_{min}
\end{equation}
\begin{figure}[h]
    \begin{subfigure}[][][1]{0.5\textwidth}
        \centering
    \resizebox{1\textwidth}{!}{%
    \begin{circuitikz}
    \tikzstyle{every node}=[font=\Large]
    
    
    \draw [ line width=0.6pt ] (8,16.5) rectangle  node {\Large VCO} (13.25,13.25);
    \draw [->, >=Stealth] (6.75,15) -- (8,15);
    \draw [->, >=Stealth] (13.25,15) -- (14.75,15);
    \node [font=\Large] at (5.5,15) {$V_{Cntrl}(V_{C})$};
    \node [font=\Large] at (15.3,15) {$f_{out}$};
    \end{circuitikz}
    }%
    \end{subfigure}
    \begin{subfigure}[][][1]{0.5\textwidth}
        \centering
            \centering
            \begin{tikzpicture}[
                axis/.style={->, thick},
                dashedline/.style={dashed, thick},
                solidline/.style={thick},
                font=\small,
                scale=1.2
            ]
            
            % Axes
            \draw[axis] (0,0) -- (6,0) node[right] {$V_C$};
            \draw[axis] (0,0) -- (0,5) node[above] {$f$};
            
            % Coordinates
            \coordinate (Vcmin) at (1,0);
            \coordinate (Vcnom) at (3,0);
            \coordinate (Vcmax) at (5,0);
            
            \coordinate (fcmin) at (0,1);
            \coordinate (fcnom) at (0,3);
            \coordinate (fcmax) at (0,4.5);
            
            \coordinate (p1) at (1,1);
            \coordinate (p2) at (3,3);
            \coordinate (p3) at (5,4.5);
            
            % Vertical dashed lines
            \draw[dashedline] (Vcmin) -- (p1);
            \draw[dashedline] (Vcnom) -- (p2);
            \draw[dashedline] (Vcmax) -- (p3);
            
            % Horizontal dashed lines
            \draw[dashedline] (fcmin) -- (p1);
            \draw[dashedline] (fcnom) -- (p2);
            \draw[dashedline] (fcmax) -- (p3);
            
            % VCO Line
            \draw[solidline] (p1) -- (p3);
            
            % Labels
            \node[below] at (Vcmin) {$V_{C(\min)}$};
            \node[below] at (Vcnom) {$V_{C(\text{nom})}$};
            \node[below] at (Vcmax) {$V_{C(\max)}$};
            
            \node[left] at (fcmin) {$f_{o(\min)}$};
            \node[left] at (fcnom) {$f_{o(\text{nom})}$};
            \node[left] at (fcmax) {$f_{o(\max)}$};
            
            \end{tikzpicture}
    \end{subfigure}            
    \caption{VCO Block Diagram and Characteristics}
    \label{fig:vco_block}
    % \vspace{-0.5cm}
\end{figure}
The VCO transfer function can be given as
\begin{equation}
	\label{eq:vco_tf}
	H_{vco}(s) = \frac{\phi_{o}(s)}{v_{o}(s)} = \frac{K_{vco}}{S}
\end{equation}
The gain of the voltage-controlled oscillator is simply the slope of the curves given in Fig.\ref{fig:vco_block}. This gain can be written as
\begin{equation}
	\label{eq:vco_gain}
	K_{vco} = 2\pi  * \frac{f_{max} - f_{min}}{V_{max} - V_{min}}(radians/s * V)
\end{equation}
Kvco is an important factor it Determines the PLL settling time.
There are different types of VCO like : Cross Coupled LC VCO, Current Starved VCO, Colpitts Oscillator. For this project, Current Starved VCO is used.
\subsection{Frequency Divider}
The frequency divider is a circuit that divides the frequency of the output signal from the VCO by a fixed integer value. The frequency divider is used to reduce the frequency of the output signal to match the frequency of the reference signal. The frequency divider can be implemented using a flip-flop or a counter. The output of the frequency divider is fed back to the phase detector to complete the PLL loop.
\subsubsection*{Frequency Divider Principle}
A frequency divider works by toggling the output state of a flip-flop at each rising (positive) edge of the input clock signal. A single flip-flop divides the frequency of the input clock by a factor of 2. Cascading multiple flip-flops results in further division:
\begin{equation}
	\label{eq:freq_div}
	f_{\text{out}} = \frac{f_{\text{in}}}{2^n}
\end{equation}
Where:
\begin{itemize}
    \item $f_{\text{in}}$ is the input clock frequency,
    \item $f_{\text{out}}$ is the output frequency after division,
    \item $n$ is the number of flip-flops connected in series.
\end{itemize}
\subsubsection*{Positive Edge-Triggered Flip-Flop}
A positive edge-triggered flip-flop changes its output state only at the rising edge of the clock signal. A T (toggle) flip-flop toggles its output on each clock edge. However, a D flip-flop can be configured to behave as a T flip-flop by connecting the inverted output back to the input:
\[
D = \sim Q
\]
This ensures the flip-flop toggles its output on each positive clock edge.
\subsubsection*{Implementation Using Pass Gates and Inverters}
A D flip-flop can be constructed using two D latches in a master-slave configuration, controlled by a clock and its complement. Each latch consists of pass gates and inverters.
\begin{itemize}
    \item Pass gates (transmission gates) are used to control the flow of data based on the clock signal. They are bidirectional switches typically made using a combination of NMOS and PMOS transistors.
    \item Inverters act as buffers and memory elements to store and propagate the logic state.
\end{itemize}
\subsubsection*{Frequency Division Design Circuit Using LT Spice}

\subsubsection*{Positive Edge-Triggered Flip-Flop}
A positive edge-triggered flip-flop changes its output state only at the rising edge of the clock signal. A T (toggle) flip-flop toggles its output on each clock edge. However, a D flip-flop can be configured to behave as a T flip-flop by connecting the inverted output back to the input:
\[
D = \sim Q
\]
This ensures the flip-flop toggles its output on each positive clock edge.
\subsubsection*{Implementation Using Pass Gates and Inverters}
A D flip-flop can be constructed using two D latches in a master-slave configuration, controlled by a clock and its complement. Each latch consists of pass gates and inverters.
\begin{itemize}
    \item Pass gates (transmission gates) are used to control the flow of data based on the clock signal. They are bidirectional switches typically made using a combination of NMOS and PMOS transistors.
    \item Inverters act as buffers and memory elements to store and propagate the logic state.
\end{itemize}
\subsubsection*{Frequency Division Design Circuit Using LT Spice}
To design a frequency divider circuit in LT Spice, follow these steps:
\begin{itemize}
    \item Create a schematic with a clock source and a D flip-flop.
    \item Configure the D flip-flop to toggle on each clock edge by connecting its inverted output back to the input.
    \item Simulate the circuit to observe the frequency division at the output.
    \item For cascading, connect the output of one flip-flop to the clock input of the next stage.
\end{itemize}
\subsubsection*{Working Principle}
\begin{itemize}
    \item When CLK = 1, the master latch is transparent and allows input data to propagate, while the slave latch holds its state.
    \item When CLK = 0, the master latch latches the input data, and the slave latch becomes transparent to pass the stored value to the output.
    \item This configuration ensures that data is transferred to the output only on the rising edge of the clock, making it a positive edge-triggered flip-flop.
\end{itemize}
\subsubsection*{Frequency Division Operation}
By cascading multiple such flip-flops, a frequency divider circuit is realized. The output of each flip-flop acts as the clock for the next stage. Since each stage toggles at half the frequency of the previous one, the overall division factor is \(2^n\).

For example:
\begin{itemize}
    \item 1 Flip-Flop → Divide by 2
    \item 2 Flip-Flops → Divide by 4
    \item 3 Flip-Flops → Divide by 8
\end{itemize}
\begin{figure}[H] % Use 'H' specifier for strict placement
    \centering
    \includegraphics[width=0.8\textwidth]{figs/circuit.png}
    \caption{Frequency Divider Circuit}
    \label{fig:LT Spice Circuit}
\end{figure}
The schematic shown in the figure represents a frequency divider circuit designed using positive edge-triggered D flip-flops. The circuit is implemented at the transistor level using pass gates (transmission gates) and inverters, which are fundamental components in CMOS logic design. This implementation provides a realistic view of how sequential circuits function at a lower abstraction level, offering better understanding of timing, logic flow, and hardware behavior.
The design uses a PULSE voltage source to generate the clock signal. This source is defined to oscillate between 0V and 1.8V with 1 ns rise and fall times, a pulse width of 100 ns, and a time period of 300 ns. The generated clock signal (clk) is used to drive the flip-flops in the circuit. An inverter is used to create the complementary clock signal (clk\_b), which is necessary to properly control the pass gates in the master-slave latch configuration of each D flip-flop.

Each flip-flop is constructed using two D latches connected in a master-slave configuration. Each latch consists of a pair of transmission gates, controlled by the clock and its complement, and inverters, which serve to store and propagate the logic state. This structure ensures that the flip-flop captures input data only on the rising edge of the clock signal, making it a positive edge-triggered flip-flop. To make the flip-flops function as T (toggle) flip-flops, the inverted output (\(Q_b\)) is fed back to the D input of each flip-flop. This feedback ensures that the flip-flop toggles its output on every rising edge of the clock.

In this circuit, two flip-flops are cascaded to form a 2-stage frequency divider. The first flip-flop toggles its output on every clock cycle, effectively dividing the input frequency by 2. The second flip-flop receives the output of the first as its clock input and toggles on every rising edge of that signal, thereby dividing the frequency by another factor of 2. As a result, the final output signal has a frequency equal to one-fourth of the original input clock. This cascading approach can be extended to more stages for greater frequency division.

The simulation is performed using SPICE, with the \texttt{.tran 100u} command specifying a transient analysis over a period of 100 microseconds to observe the dynamic behavior of the circuit. The schematic also includes a reference to a process design kit (\texttt{.INCLUDE tsmc018.lib}), which models the behavior of transistors based on the TSMC 180nm CMOS technology. This provides accurate transistor-level simulation results, allowing verification of correct operation and timing characteristics.

In conclusion, the schematic demonstrates the design and working of a CMOS-based frequency divider using positive edge-triggered D flip-flops implemented with pass gates and inverters. It effectively divides the clock frequency by powers of two and can be used in a variety of digital systems requiring timing control, clock scaling, or counter functionality.

\begin{figure}[H] % Use 'H' specifier for strict placement
    \centering
    \includegraphics[width=0.8\textwidth]{figs/waveform.png}
    \caption{Frequency Divider Output Waveform}
    \label{fig:Output Waveform}
\end{figure}
\subsubsection*{Conclusion}
In this project, a frequency divider circuit was successfully designed and implemented using positive edge-triggered D flip-flops constructed with pass gates and inverters. The use of transistor-level design under TSMC 180nm CMOS technology provided a deeper understanding of the internal working of sequential circuits. Simulation results verified that each flip-flop stage effectively divides the input clock frequency by a factor of two. This design is efficient for low-power, high-speed digital systems and demonstrates the practical use of basic building blocks in creating complex timing circuits.

\subsubsection*{Applications}
\begin{itemize}
    \item \textbf{Clock Division in Microprocessors:} Used to generate lower frequency clocks from a high-speed master clock.
    \item \textbf{Digital Watches and Timers:} Essential for time base generation.
    \item \textbf{Pulse Generation Circuits:} Helps in creating precise timing pulses for control systems.
    \item \textbf{Frequency Synthesizers:} Used in communication systems to derive required frequencies.
\end{itemize}
\section{Design of PLL blocks}
The numberous block have been designed for the PLL taking in consideration of the practical aspects.
\subsection{PFD Design}
As shown in figure \ref{fig:pfd_dff_blk}, PFD circuit requires two D flip-flops, and one AND gate. A very fast RESET is required in order to reduce the PFD dead-zone.Hence it limits the reference frequency at which the pll can be clocked at.Here to overcome the dead zone issue we have implemented a nand based pfd.
% \begin{figure}
% 	\centering
% 	\includegraphics[width=0.5\textwidth]{figs/pfd_dff_blk.png}
% 	% \vspace{-0.3cm}
% 	\caption{PFD circuit using D flip-flops}
% 	\label{fig:pfd_dff_blk}
% 	\vspace{0.5cm}
% \end{figure}
\subsubsection{2 intput NAND gate}
The 2 input NAND gate is implement in static CMOS logic. The circuit is made using 4 transistors, 2 PMOS and 2 NMOS. The PMOS transistors are connected in parallel and the NMOS transistors are connected in series. The output of the NAND gate is taken from the drain of the NMOS transistors.
\subsubsection{3 input NAND gate}
\subsubsection{4 input NAND gate}
\subsubsection{inverter}
\subsubsection{NAND based PFD}
\subsection{Charge Pump Design}
\subsection{Low Pass Filter Design}
\subsection{Voltage Controlled Oscillator Design}
In this section the Design of VCO has been shown.We have choosen current starved VCO for our design. The current starved VCO is a type of voltage-controlled oscillator (VCO) that uses a current source to control the frequency of oscillation. The basic idea behind the current starved VCO is to use a current source to control the charging and discharging of a capacitor, which in turn determines the frequency of oscillation. The current starved VCO is widely used in PLL circuits because it is simple to implement and can be easily integrated into CMOS technology.
The VCO should be Linear in a particular Operating region. The PLL built in this project will be of of the application 1GHz


% \includegraphics[0.6\textwidth]{figs/cs_vco_design.png}
\begin{figure}[h]
	\centering
	\includegraphics[width=0.5\textwidth]{figs/cs_vco_design.png}
	% \vspace{-0.3cm}
	\caption{Current Starved VCO Design}
	\label{fig:cs_vco_design}
	\vspace{0.5cm}
\end{figure}
\begin{figure}[h]
	\centering
	\includegraphics[width=0.5\textwidth]{figs/vco_simplified.png}
	% \vspace{-0.3cm}
	\caption{Simplified view of a single stage of the current-starved VCO}
	\label{fig:vco_simplified}
\end{figure}

To determine the design equations for use with the current-starved VCO, consider
the simplified schematic of one stage of the VCO fig \ref{fig:vco_simplified}. The total
capacitance on the drains of M2 and M3 is given by
\begin{equation}
	C_{\text{total}} = C_{\text{out}} + C_{\text{in}} = 
\underbrace{C'_{\text{ox}}(W_p L_p + W_n L_n)}_{\text{C\textsubscript{out}}} + 
\underbrace{\frac{3}{2} C'_{\text{ox}}(W_p L_p + W_n L_n)}_{\text{C\textsubscript{in}}}
\end{equation}
which is simply the output and input capacitances of the inverter. This equation can be written in a more useful form as
\begin{equation}
C_{\text{tot}} = \frac{5}{2} C'_{\text{ox}} (W_p L_p + W_n L_n)
\tag{19.19}
\end{equation}

\noindent The time it takes to charge $C_{\text{total}}$ from zero to $V_{SP}$ with the constant-current $I_{D4}$ is given by
\begin{equation}
t_1 = C_{\text{tot}} \cdot \frac{V_{SP}}{I_{D4}}
\tag{19.20}
\end{equation}

\noindent while the time it takes to discharge $C_{\text{total}}$ from $V_{DD}$ to $V_{SP}$ is given by
\begin{equation}
t_2 = C_{\text{tot}} \cdot \frac{V_{DD} - V_{SP}}{I_{D1}}
\tag{19.21}
\end{equation}

If we set $I_{D4} = I_{D1} = I_D$ (which we will label $I_{\text{Dcenter}}$ when $V_{\text{inVCO}} = V_{DD}/2$), then the sum of $t_1$ and $t_2$ is simply
\begin{equation}
t_1 + t_2 = \frac{C_{\text{tot}} \cdot V_{DD}}{I_D}
\tag{19.22}
\end{equation}

The oscillation frequency of the current-starved VCO for $N$ (an odd number $\geq 5$) of stages is
\begin{equation}
f_{\text{osc}} = \frac{1}{N(t_1 + t_2)} = \frac{I_D}{N \cdot C_{\text{tot}} \cdot V_{DD}}
\tag{19.23}
\end{equation}

which is $f_{\text{center}}(@ V_{\text{inVCO}} = V_{DD}/2 \text{ and } I_D = I_{\text{Dcenter}})$

firstly we have designed the Inverter stages and sized them accordingly and cascaded them to our required and connected a current Mirror to all the stages as in the Fig \ref{fig:vco_circuit}. The current mirror is used to control the current flowing through the inverter stages and thus control the frequency of oscillation. The output of the VCO is taken from the output of the last inverter stage. The VCO is designed to operate at a frequency of 1 GHz.
\begin{figure}[h]
	\centering
	\includegraphics[width=0.9\textwidth]{figs/vco_c.png}
	% \vspace{-0.3cm}
	\caption{Current Starved VCO Circuit}
	\label{fig:vco_circuit}
	\vspace{0.5cm}
\end{figure}\\
The VCO ouput is not inherently a square wave (refer fig:\ref{fig:vco_op_c}) due to non-Ideal charachteristic of inverter here the ouput of the VCO is a irregular triangular wave. The output of the VCO is fed to a buffer and an inverter to convert the  of triangular wave to a square wave.
\begin{figure}
	\centering
	\includegraphics[width=0.9\textwidth]{figs/vco_op_both.png}
	% \vspace{-0.3cm}
	\caption{VCO output waveform before and after buffer+inverter}
	\label{fig:vco_op_c}
	\vspace{0.5cm}
\end{figure}
In Figure \ref{fig:vco_sim}, the Transient analysis of circuit in Figure \ref{fig:vco_circuit} is shown.
\begin{figure}[H]
	\centering
	\includegraphics[width=0.9\textwidth]{figs/vco_op1.png}
	% \vspace{-0.3cm}
	\caption{output waveform of VCO}
	\label{fig:vco_sim}
\end{figure}

\subsection{Frequency Divider Design}
\label{sec:freq_div}
\subsection{Complete PLL Design}
\label{sec:complete_pll}



  % \chapter{Results}
\label{chapter:results}

We obtain results for our two sub-tasks on selected popular datasets. The results are reported using standard metrics commonly used to evaluate these tasks. 


\section{Multi Object Tracking Evaluation}

\subsection{Datasets and Evaluation metrics}
Experiments are conducted on the MOT16 \cite{DeepSiam:MilanL0RS16} and KITTI \cite{DeepSiam:KITTI} tracking datasets. The MOT16 dataset contains 7 videos in its training set. The KITTI tracking dataset contains 21 videos in its training set. The Siamese Network for appearance consistency is trained completely on external data (ImageNet datasets) and there is no overlap with any of the MOT16 or KITTI data. The LSTM network is trained only with the use of bounding box locations of objects and class information for a partition of the training sets of these two datasets (the remainder is kept aside for testing purposes). Results are reported for our test partition (in the case of LSTM usage) and for the entire datasets (in cases they are not used for training).
\par Evaluation of our system is carried out for the entire system as well as for the study of LSTM network alone. For the case of the entire system, we consider the metrics used by the MOT benchmarks for evaluation. This includes Multiple Object Tracking Accuracy (MOTA), Multiple Object Tracking Precision (MOTP), the ratio of Mostly Tracked targets (MT), and the ratio of Mostly Lost targets (ML). In the case of the LSTM network, the Average Precision (AP) value for the predicted frames across the dataset and classes is reported.

\subsection{Evaluation}
The evaluations on the MOT16 Dataset for the end to end system are reported in Table I. Evaluations mainly focus on two aspects: improvement in accuracy with the introduction of the similarity measure to a traditional tracker using only a Kalman filter or an LSTM network and how closely related the accuracy is with state of the art multi-object trackers. Similar results on the KITTI tracking dataset are presented for our work alongside comparisons (note that few state-of-the-art works report on this dataset) in Table II. Separate evaluations for the LSTM in the case of single object tracking for individual tracklets in the KITTI dataset was carried out. An average IoU of 61.45 and AP of 0.96 at 0.5 IoU were obtained for this experiment.

\begin{table*}[ht]
	\caption[Results on MOT Dataset]{Comparison  of  our  performance  on  MOT16  dataset  with  recent  works}
	\label{MOT16 Results}
	\begin{center}
		\begin{tabular}{|c||c||c||c||c||c|}
			\hline
			Method & Mode & MOTA$\uparrow$ & MOTP$\uparrow$ & MT$\uparrow$ & ML$\downarrow$\\
			\hline
			Deep SORT \cite{DeepSiam:deepSort} & ONLINE & 61.40\% & 79.10\% & 32.80\% & 18.20\%\\
			\hline
			SORT \cite{DeepSiam:Sort} & ONLINE & 59.80\% & 79.60\% & 25.40\% & 22.70\%\\
			\hline
			RNN LSTM \cite{DeepSiam:multitarget} & ONLINE & 19.00\% & 71.00\% & 05.50\% & 45.60\%\\
			\hline
			MDP \cite{DeepSiam:learntotrack} & ONLINE & 30.30\% & 71.30\% & 13.00\% & 38.40\%\\
			\hline
			DMAN \cite{DeepSiam:attention} & ONLINE & 46.10\% & 73.80\% & 17.40\% & 42.70\%\\
			\hline
			LSTM+Similarity (Ours) & ONLINE & 66.70\% & 69.00\% & 39.18\% & 16.80\%\\
			\hline
			Kalman Filter (Ours) & ONLINE & 61.00\% & 69.00\% & 17.00\% & 17.00\%\\
			\hline
		\end{tabular}
	\end{center}
\end{table*}

\begin{table*}[h]
	\caption[Results on KITTI dataset]{Comparison  of  our  performance  on  KITTI-trracking  dataset  with  recent  works}
	\label{KITTI Results}
	\begin{center}
		\begin{tabular}{|c||c||c||c||c||c|}
			\hline
			Method & Mode & MOTA$\uparrow$ & MOTP$\uparrow$ & MT$\uparrow$ & ML$\downarrow$\\
			\hline
			Regionlets Only \cite{DeepSiam:kalman} & ONLINE & 76.40\% & 81.50\% & 54.10\% & 9.30\%\\
			\hline
			MS-CNN Only \cite{DeepSiam:kalman} & ONLINE & 81.23\% & 85.60\% & 66.30\% & 4.60\%\\
			\hline
			Regionlets MS-CNN \cite{DeepSiam:kalman} & ONLINE & 82.60\% & 85.00\% & 70.50\% & 5.30\%\\
			\hline
			SMES \cite{DeepSiam:simmap} & ONLINE & 70.78\% & 80.38\% & 51.68\% & 7.77\%\\
			\hline
			LSTM + Similarity (Ours) & ONLINE & 83.58\% & 78.50\% & 48.23\% & 2.25\%\\
			\hline
		\end{tabular}
	\end{center}
\end{table*}


\subsection{Experiments to analyze the extensibility of the modules}

LSTM based data association for end to end trainability: The LSTM network was trained under negative log likelihood loss. It was observed that network was not developing a significant convergence even for a fixed set of data associations which was also the key expectation. This methodology is trainable but in comparison to the results from the Hungarian algorithm, the associations are sub-optimal and have a significant potential of resulting in non-coherent results (similar to observations at the training phase) which deprecate the accuracy of the entire system henceforth. The results being inconsistent as well as non-coherent and the observation that training sessions do not converge to a feasible setting made this sub module unsuccessful in terms of performance.

Feature Predictor: The feature space is significant in every novel approach considered in the fields of vision based analysis where tracking is only a sub group of it. Along with the ability of the LSTM networks to perform well in prediction, and as model predictors used in current trackers perform linear interpolation of the feature tensors, an experiment on the ability of a trainable network to predict the feature space was conducted (generic interpolation of features is done in most of the Siamese tracking networks and they are proven to perform well in practice).
During this analysis, a robust LSTM network with high hidden state size was trained on the extracted features in sequences. (Ability to compare a complete feature vector can be integrated for optical flow analysis and many further approaches if turns out to be successful). However the feature predicting network was over fitting to the dataset (custom) during the training phase. The accuracy of the tests for network validation was poor and turned out insufficient for any further analysis.


\section{Panoptic Segmentation Evaluation}

In this section, we first show the convergence of the mean field based inference algorithm for BCRF and then show the usefulness of the BCRF model by evaluating its performance on the Pascal VOC dataset and the COCO dataset.

\subsection{Convergence of the Inference}
\begin{figure}[t]
\vspace{-0.7cm}
  \centering
  \includegraphics[ width=0.5\textwidth]{figs/kl_div.png}
  \vspace{-0.3cm}
  \caption[Convergence of BCRF Inference]{{\bf Convergence of BCRF Inference.} Convergence of KL divergence with the number of iterations.\vspace{0.0cm}}
  \label{fig:kl_div}
\vspace{0.5cm}
\end{figure}

It is difficult to provide a theoretical convergence guarantee for mean field algorithms with parallel updates~\cite{Higherorder_mf, Koller_book}. We therefore provide empirical evidence to show that the presented mean field inference algorithm for our BCRF with cross potentials converge under normal conditions. To this end, we estimate the KL divergence between the original joint distribution and the factorized distribution (see Eq.~\eqref{eq:q_approx}), at the end of each iteration in Algorithm~\ref{alg:infer}. Note that this KL divergence can be estimated up to a constant using the method described in ~\cite{densecrf_suppl}. We pick 20 random images from the Pascal VOC validation set and average the KL divergence for each iteration across these images. The resulting plot is shown in Fig.~\ref{fig:kl_div}. It can be seen that the KL divergence measure, and therefore the inference algorithm, converges within a few iterations. We also note that visual results do not change after about 5 iterations.

\subsection{Bipartite Potentials Learning}
Figure \ref{fig:potentials} illustrates how important logits belonging to each class in the instance branch are for predicting each class in the semantic branch when the model has been fully trained. Our BCRF module allows the network to learn complex relationships between the semantic and instance features belonging to each class. While there is room for it to learn a simple logical relationship, the variation of learned parameters in Figure \ref{fig:potentials} verifies that a complex class-specific mapping has been learned by the network. 

\subsection{Results on the Pascal VOC Dataset}

In this experiment we use the architecture shown in Figure~\ref{fig:bcrf_net} and CNN components similar to the ones used in~\cite{Upsnet_paper}. More specifically, we use a ResNet-50 with an FPN as the backend, to which we attach a fully convolutional network as the semantic segmentation head and a Mask R-CNN network as the instance segmentation head. 

During both training and inference we used 5 mean-field iterations for BCRF. At the output, we calculate the loss function as a summation of two components: the usual pixel-wise categorical cross entropy loss for the semantic component~\cite{FCN_2015} and the loss used in~\cite{Anurag17} for the instance component. We used full-image training with batch size 1 and SGD with learning rate $0.0007$ and momentum $0.99$. In Table~\ref{tbl:pascal_e2e}, we report the summary of the quantitative results. Table~\ref{tbl:pascal_detailed} shows the class-wise results. Qualitative results are shown in Table~\ref{tbl:pascal_visual}, where benefits of optimally combining the semantic segmentation classification and instance segmentation classification with BCRF can be seen.

\begin{table}[]\centering
	\begin{tabular}{|c|c|c|c|}
		\hline
		\textbf{Method} & \textbf{PQ} & \textbf{SQ} & \textbf{RQ} \\ \hline
		DeeperLab \cite{deeperLab}      & 67.35       & -           & -           \\ \hline
		Ours (baseline) & 70.50       & 88.65       & 78.83       \\ \hline
		Ours (CRF only) & 67.72       & 87.62       & 76.48       \\ \hline
		Ours (BCRF)     & 71.76       & 89.63       & 79.33       \\ \hline
	\end{tabular}
	\vspace{-0.2cm}
	\caption[Comparison of results on Pascal VOC dataset]{Comparison of results on Pascal VOC dataset. The baseline used contains DeepLab-v3 for semantic branch and Mask-RCNN for instance branch followed by combination using the simple logical method outlined in \cite{panoptickirillov2017}. CRF only corresponds to setting the BCRF cross-potential terms to zero. BCRF is our complete network.}
	\label{tbl:pascal_e2e}
	\vspace{0.5cm}
\end{table}
\begin{table*}[t]
	\vspace{-0.2cm}
	\hspace{0.2cm}
	\centering
	{\arraybackslash
\begin{tabular}{l|c|c|c|c|c|c}
	\hline
	\textbf{}	         & \multicolumn{2}{c|}{\textbf{PQ}} & \multicolumn{2}{c|}{\textbf{SQ}} & \multicolumn{2}{c}{\textbf{RQ}}  \\ \hline
	\textbf{Class}       & W/O BCRF       & BCRF            & W/O BCRF        & BCRF           & W/O BCRF        & BCRF           \\ \hline
	\textbf{Background}  & 90.8           & 92.33           & 93.39           & 94.69          & 97.22           & 97.51          \\ \hline
	\textbf{Aeroplane}   & 78.55          & 80.37           & 88.57           & 92.6           & 88.68           & 86.79          \\ \hline
	\textbf{Bicycle}     & 29.78          & 31.71           & 67.36           & 68.46          & 44.21           & 46.32          \\ \hline
	\textbf{Bird}        & 84.98          & 85.09           & 93.05           & 93.24          & 91.32           & 91.25          \\ \hline
	\textbf{Boat}        & 65.83          & 66.21           & 85.33           & 86.48          & 77.14           & 76.56          \\ \hline
	\textbf{Bottle}      & 67.44          & 64.05           & 92.05           & 90.68          & 73.26           & 70.63          \\ \hline
	\textbf{Bus}         & 82.68          & 82.58           & 94.56           & 95.46          & 87.44           & 86.51          \\ \hline
	\textbf{Car}         & 72.22          & 70.93           & 93.69           & 91.7           & 77.08           & 77.35          \\ \hline
	\textbf{Cat}         & 77.41          & 83.4            & 91.24           & 93.73          & 84.85           & 88.97          \\ \hline
	\textbf{Chair}       & 43.3           & 41.79           & 82.5            & 82.64          & 52.49           & 50.57          \\ \hline
	\textbf{Cow}         & 76.91          & 80.42           & 92.81           & 93.95          & 82.87           & 85.6           \\ \hline
	\textbf{Diningtable} & 51.33          & 51.8            & 80.81           & 82.88          & 63.51           & 62.5           \\ \hline
	\textbf{Dog}         & 76.63          & 81.59           & 90.5            & 93.29          & 84.67           & 87.46          \\ \hline
	\textbf{Horse}       & 76.86          & 81.4            & 89.38           & 91.11          & 86              & 89.34          \\ \hline
	\textbf{Motorbike}   & 78.07          & 80.21           & 87.5            & 89.89          & 89.23           & 89.23          \\ \hline
	\textbf{Person}      & 76.33          & 77              & 89.75           & 89.73          & 85.05           & 85.81          \\ \hline
	\textbf{Pottedplant} & 58.98          & 60.62           & 85.41           & 85.32          & 69.06           & 71.05          \\ \hline
	\textbf{Sheep}       & 74.29          & 74              & 93.86           & 93.48          & 79.15           & 79.15          \\ \hline
	\textbf{Sofa}        & 60.37          & 62.12           & 88.47           & 89.5           & 68.24           & 69.41          \\ \hline
	\textbf{Train}       & 78.52          & 80.05           & 88.7            & 90.43          & 88.52           & 88.52          \\ \hline
	\textbf{Tvmonitor}   & 79.23          & 79.34           & 92.8            & 92.93          & 85.38           & 85.38          \\ \hline
	\textbf{Mean Value}  & \textbf{70.5}  & \textbf{71.76}  & \textbf{88.65}  & \textbf{89.63} & \textbf{78.83}  & \textbf{79.33} \\ \hline
\end{tabular}
	}
	\vspace{0.0cm}
	\caption[Detailed results on the Pascal VOC dataset]{\label{tbl:pascal_detailed}{\bf Pascal VOC dataset.} Detailed class-wise panoptic segmentation results on the Pascal VOC validation set comparing results without BCRF vs with BCRF on a standard network.}
	\vspace{0.5cm}
\end{table*}


\begin{table*}[h!]
\vspace{-0.8cm}
%  \centering
	\begin{subtable}{\textwidth}
        \centering
        {\renewcommand{\arraystretch}{0.7}
                                \begin{tabular}{ c@{\hspace{3pt}} c@{\hspace{3pt}} c@{\hspace{3pt}} c@{\hspace{3pt}} c}				
						\includegraphics[width=0.19\textwidth]{figs/pascal/original/144.jpg}%
						& \includegraphics[width=0.19\textwidth]{figs/pascal/before_sem/144.png}%
						& \includegraphics[width=0.19\textwidth]{figs/pascal/before_ins/144.png}%
						& \includegraphics[width=0.19\textwidth]{figs/pascal/after_sem/144.png}%
						& \includegraphics[width=0.19\textwidth]{figs/pascal/after_ins/144.png}\\
														\newline
						\includegraphics[width=0.19\textwidth]{figs/pascal/original/421.jpg}%
						& \includegraphics[width=0.19\textwidth]{figs/pascal/before_sem/421.png}%
						& \includegraphics[width=0.19\textwidth]{figs/pascal/before_ins/421.png}%
						& \includegraphics[width=0.19\textwidth]{figs/pascal/after_sem/421.png}%
						& \includegraphics[width=0.19\textwidth]{figs/pascal/after_ins/421.png}\\
						                                \newline
						\includegraphics[width=0.19\textwidth]{figs/pascal/original/438.jpg}%
						& \includegraphics[width=0.19\textwidth]{figs/pascal/before_sem/438.png}%
						& \includegraphics[width=0.19\textwidth]{figs/pascal/before_ins/438.png}%
						& \includegraphics[width=0.19\textwidth]{figs/pascal/after_sem/438.png}%
						& \includegraphics[width=0.19\textwidth]{figs/pascal/after_ins/438.png}\\
						                               \newline
						\includegraphics[width=0.19\textwidth]{figs/pascal/original/468.jpg}%
						& \includegraphics[width=0.19\textwidth]{figs/pascal/before_sem/468.png}%
						& \includegraphics[width=0.19\textwidth]{figs/pascal/before_ins/468.png}%
						& \includegraphics[width=0.19\textwidth]{figs/pascal/after_sem/468.png}%
						& \includegraphics[width=0.19\textwidth]{figs/pascal/after_ins/468.png}\\
														\newline
						\includegraphics[width=0.19\textwidth]{figs/pascal/original/184.jpg}%
                        & \includegraphics[width=0.19\textwidth]{figs/pascal/before_sem/184.png}%
                        & \includegraphics[width=0.19\textwidth]{figs/pascal/before_ins/184.png}%
                        & \includegraphics[width=0.19\textwidth]{figs/pascal/after_sem/184.png}%
                        & \includegraphics[width=0.19\textwidth]{figs/pascal/after_ins/184.png}\\
                                                        \newline
						\includegraphics[width=0.19\textwidth]{figs/pascal/original/226.jpg}%
                        & \includegraphics[width=0.19\textwidth]{figs/pascal/before_sem/226.png}%
                        & \includegraphics[width=0.19\textwidth]{figs/pascal/before_ins/226.png}%
                        & \includegraphics[width=0.19\textwidth]{figs/pascal/after_sem/226.png}%
                        & \includegraphics[width=0.19\textwidth]{figs/pascal/after_ins/226.png}\\
%                                                          \newline
%  						\includegraphics[width=0.19\textwidth]{figs/pascal/original/233.jpg}%
%                          & \includegraphics[width=0.19\textwidth]{figs/pascal/before_sem/233.png}%
%                          & \includegraphics[width=0.19\textwidth]{figs/pascal/before_ins/233.png}%
%                          & \includegraphics[width=0.19\textwidth]{figs/pascal/after_sem/233.png}%
%                          &\includegraphics[width=0.19\textwidth]{figs/pascal/after_ins/233.png}\\														
% 														
                                \end{tabular}
                                }
    \end{subtable}
    \vspace{0.0cm}
    \caption[Visualizations on Pascal VOC]{{\bf Visualizations on Pascal VOC.} Example images from the Pascal VOC validation set. Columns left to right: original image, semantic output before BCRF, instance output before BCRF, semantic output after BCRF, instance output after BCRF. Each row contains a new image. The standard Pascal VOC color map is used for the semantic segmentation results.}
    \label{tbl:pascal_visual}
    \vspace{0.1cm}
\end{table*}


\subsection{Results on the COCO Dataset}
To further evaluate the usefulness of BCRF without any efforts for end-to-end training, experiments were conducted on the COCO dataset by simply plugging in the BCRF on an existing pre-trained model. We used a combination of publicly available models of~\cite{Upsnet_paper, object_detection_api}, which produced a PQ score of 41.4\% on the COCO validation set. The parameters of the BCRF were hand-tuned using a small subset of train images. Results obtained from that BCRF model without end-to-end training are listed in Table~\ref{tbl:coco_values}.

\begin{table*}[t]
\vspace{-0.2cm}
\hspace{0.2cm}
\centering
{\arraybackslash
\begin{tabular}{c|c|c|c|c|c|c|c}
	\hline
	& \multicolumn{2}{c|}{\textbf{PQ}} & \multicolumn{2}{c|}{\textbf{SQ}} & \multicolumn{2}{c|}{\textbf{RQ}} &         \\ \hline
	\textbf{Category} & W/O BCRF          & BCRF         & W/O BCRF          & BCRF         & W/O BCRF          & BCRF         & Classes \\ \hline
	\textbf{All}      & 41.4              & 41.7         & 78.3              & 79.1         & 50.8              & 51.1         & 133     \\
	\textbf{Things}   & 47.4              & 47.4         & 80.4              & 80.4         & 57.3              & 57.3         & 80      \\
	\textbf{Stuff}    & 32.5              & 33.2         & 75.1              & 77.1         & 40.9              & 41.6         & 53      \\ \hline
\end{tabular}
}
\vspace{0.0cm}
\caption[Results on COCO dataset]{\label{tbl:coco_values}{\bf COCO dataset.} Panoptic segmentation results on the COCO validation set.}
\vspace{0.5cm}
\end{table*}




\begin{figure}[t]
	\begin{center}
		\includegraphics[width=\linewidth]{figs/stuff_vis.png}
	\end{center}
	\vspace{-0.5cm}
	\caption{Visualisation of improvements on COCO Dataset}
	\label{fig:vis}
	\vspace{0.5cm}
\end{figure}

\subsection{BCRF learns beyond simple logical mapping}
Figure \ref{fig:potentials} illustrates how important logits belonging to each class in the instance branch are for predicting each class in the semantic branch when the model has been fully trained. Our BCRF module allows the network to learn complex relationships between the semantic and instance features belonging to each class. While there is room for it to learn a simple logical relationship, the variation of learned parameters in Figure \ref{fig:potentials} verifies that a complex class-specific mapping has been learned by the network. 

\begin{figure}[t]
	\begin{center}
		\includegraphics[width=0.8\linewidth]{figs/ins_to_sem.png}
	\end{center}
	\vspace{-0.4cm}
	\caption[Bipartite Potentials Visualization]{The heatmap illustrates inter-class dependencies learned by the cross-potential term weights of BCRF. Note that a logarithmic scale has been used. }
	\label{fig:potentials}
	\vspace{0.5cm}
\end{figure}
  % \chapter{Discussion and Conclusion}

The goal of this Project was to develop and simulate Transistor Level Circuit of a PLL which can be used in implementing in RFICs.The PLL has been designed using 180nm CMOS technology and the circuit has been simulated using LTspice.All these circuits where designed and optimized to get a good settling time.There is huge scope of Imporvement in the same with this project we have showcased the proof of concept of the PLL and its working.

  \singlespacing
  \include{chapters/References}
  \chapter*{Appendix I}
\addcontentsline{toc}{chapter}{Appendix I}
\section*{Appendix I: Circuit Schematic}
\addcontentsline{toc}{section}{Circuit Schematic}	
\chapter*{Appendix II}

\chapter*{Appendix III}

  \documentclass[12pt]{article}
\usepackage{amsmath,amssymb,graphicx}
\usepackage{float}
\usepackage{caption}
\usepackage{geometry}
\geometry{a4paper, margin=1in}

\title{Charge Pumps}
\author{}
\date{}

\begin{document}

\maketitle

\section{Introduction to Charge Pumps}
A charge pump is a circuit that sources or sinks charge for a controlled amount of time. A simple realization is: if $S_1$ is on, $I_1$ charges $C_1$, and if $S_2$ is on, $I_2$ discharges it.

\begin{figure}[H]
    \centering
    \includegraphics[width=0.6\textwidth]{figs/cp1}
    \caption{Basic charge pump circuit}
\end{figure}

The controls are called \textbf{Up} and \textbf{Down}, respectively, determining whether the output voltage rises or falls. Assume $I_1 = I_2 = I_p$. The transistor-level implementation uses $M_1$ and $M_2$ as current sources and $M_3$ and $M_4$ as switches. This is called a “drain-switched” CP.

\section{CP/Capacitor Cascade}
Each time a phase comparison is made, $QA$ goes high, $S_1$ turns on, $I_1$ charges $C_1$, and $V_{out}$ rises by:
\[
\Delta V = \left(\frac{I_p}{C_1}\right)\Delta T
\]

\begin{figure}[H]
    \centering
    \includegraphics[width=0.6\textwidth]{figs/cp2}
    \caption{CP/Capacitor cascade operation}
\end{figure}

If the phase difference remains constant, $V_{out}$ increases indefinitely, implying infinite gain for finite phase error. For $V_{out}$ to be finite, the phase error must be zero.

\section{Basic Charge-Pump PLL}
We now construct a PLL using the PFD/CP/capacitor cascade. If the loop is locked:
\[
f_{out} = f_{in}, \quad \phi_{out} = \phi_{in}
\]

\begin{figure}[H]
    \centering
    \includegraphics[width=0.7\textwidth]{figs/cp3}
    \caption{PLL using PFD/CP/capacitor cascade}
\end{figure}

The loop locks with zero static phase error due to the integration effect of the CP and capacitor.

\section{CP Transfer Function}
The impulse response is derived by first applying a phase step. The voltage change each cycle is:
\[
\Delta V = \left(\frac{\Delta \phi_1}{2\pi}\right)T_{in}\left(\frac{I_p}{C_1}\right)
\]
The slope is:
\[
\frac{\Delta V}{T_{in}} = \left(\frac{\Delta \phi_1}{2\pi}\right)\left(\frac{I_p}{C_1}\right)
\]
Thus, the impulse response is:
\[
h(t) = \left(\frac{\Delta \phi_1}{2\pi}\right)\left(\frac{I_p}{C_1}\right)u(t)
\]
And the Laplace transform (transfer function):
\[
\frac{V_{cont}(s)}{\Delta \phi(s)} = \frac{I_p}{2\pi C_1 s}
\]

\begin{figure}[H]
    \centering
    \includegraphics[width=0.65\textwidth]{figs/cp4}
    \caption{Impulse response analysis of charge pump}
\end{figure}

This is an ideal integrator. Combined with a VCO having $K_{VCO}/s$, the open-loop transfer function is:
\[
H_{open}(s) = \frac{I_p K_{VCO}}{2\pi C_1 s^2}
\]
This has two poles at origin (type-II PLL), and the closed-loop transfer function is:
\[
H(s) = \frac{I_p K_{VCO}}{2\pi C_1 s^2 + I_p K_{VCO}}
\]

\section{Stabilizing the Loop}
To stabilize, a resistor $R_1$ is added in series with $C_1$:
\[
\frac{V_{cont}(s)}{\Delta \phi(s)} = \frac{I_p}{2\pi} \left( \frac{1}{C_1 s} + R_1 \right)
\]

\begin{figure}[H]
    \centering
    \includegraphics[width=0.65\textwidth]{figs/cp5}
    \caption{Stabilized loop with series resistor}
\end{figure}

The loop filter introduces a zero at $s = -1/(R_1 C_1)$. The closed-loop transfer function becomes:
\[
H(s) = \frac{I_p K_{VCO}}{2\pi C_1 (R_1 C_1 s + 1) s^2 + I_p K_{VCO} R_1 s + \frac{I_p K_{VCO}}{2\pi C_1}}
\]

With:
\[
\omega_n = \sqrt{\frac{I_p K_{VCO}}{2\pi C_1}}, \quad \zeta = \frac{R_1}{2} \sqrt{\frac{2\pi}{I_p K_{VCO} C_1}}
\]

\section{Real-World Considerations}
Charge pumps exhibit ripple on control voltage requiring additional capacitance.

\begin{figure}[H]
    \centering
    \includegraphics[width=0.65\textwidth]{figs/cp6}
    \caption{Skew and ripple effects in CP}
\end{figure}

Due to skew in control signals, current imbalance results in jumps in $V_{cont}$:
\[
\Delta V = \pm I_p R_1
\]

These jumps modulate the VCO:
\[
V_{out}(f) = j\frac{K_{VCO} I_p R_1 T_{sk} \Delta T}{T_{in}} \sum_{k=-\infty}^{\infty} \delta(f - M f_{in} - k f_{in})
\]

\section{Loop Filter Impedance and PM}
Add a second capacitor $C_2$. Impedance becomes:
\[
Z(s) = \left[R_1 + \frac{1}{C_1 s} \right] \parallel \left[ \frac{1}{C_2 s} \right]
\]

\begin{figure}[H]
    \centering
    \includegraphics[width=0.6\textwidth]{figs/cp7}
    \caption{Loop filter with additional capacitor}
\end{figure}

\[
T(s) = \frac{I_p}{2\pi} \cdot \frac{R_1 C_1 s + 1}{R_1 C_{eq} s + 1} \cdot \frac{1}{(C_1 + C_2)s} \cdot \frac{K_{VCO}}{Ms}
\]

Where $C_{eq} = \frac{C_1 C_2}{C_1 + C_2}$ and third pole:
\[
\omega_{p3} = \frac{1}{R_1 C_{eq}}
\]

\begin{figure}[H]
    \centering
    \includegraphics[width=0.6\textwidth]{figs/cp8}
    \caption{Phase margin response}
\end{figure}

Phase margin degradation:
\[
PM = \tan^{-1}(R_1 C_1 \omega_u) - \tan^{-1}(R_1 C_{eq} \omega_u)
\]

\section{Second-Order Filter Topology}
A second-order filter allows $V_{R_1}$ to jump, while $C_2$ smooths ripple.

\begin{figure}[H]
    \centering
    \includegraphics[width=0.65\textwidth]{figs/cp9}
    \caption{Second-order filter topology}
\end{figure}

If $C_1 \gg C_2$, $V_{cont}$ changes minimally during CP pulse. Then $C_2$ shares charge with $C_1$, stabilizing the loop.

\begin{figure}[H]
    \centering
    \includegraphics[width=0.6\textwidth]{figs/cp10}
    \caption{Charge sharing and pulse formation}
\end{figure}

\end{document}


\end{document}
